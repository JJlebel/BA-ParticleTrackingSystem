\chapter*{Anhang}
 
%\Large{\textbf{Inhalt der beigefügten USB-Stick}}
 
\begin{tabularx}{\textwidth}{Xr}
\multicolumn{2}{c}{\emph{Inhalt der beigefügten USB-Stick}}\\
\addlinespace[1em]

Software (Partikel-Erkenung-System) \dotfill &  \pageref{kap6}\\

Vergleich zwischen Bild mit 72dpi und eins mit 30dpi \dotfill  &   \pageref{fig:kap0_72dpivs30dpi}\\

Rohbild mit zu erkennenden Partikeln \dotfill  &   \pageref{fig:kap1_rohbild}\\

Unzureichend beschriebene Optionsnamen \dotfill  &   \pageref{fig:kap1_PT_Nicht_Intuitiv}\\

Beispiel einer Partikelerkennung mit der Trackpy-Methode von ParticleTracker \dotfill  &   \pageref{fig:kap1_PT_Beispiel}\\

Beispiel von Partikelerkennung mit Trackpy \dotfill  &   \pageref{fig:kap1_Trackpy_beispiel}\\

Vergleich zwischen max iterations=1 und max iterations=1000 auf demselben Frame \dotfill  &   \pageref{fig:comparison max-iterations}\\

locate()-Funktion auf 0. Frame mit diameter=3 \dotfill  &   \pageref{fig:kap3_d=3}\\

locate()-Funktion auf 0. Frame mit diameter=9 \dotfill  &   \pageref{fig:kap3_d=9}\\

locate()-Funktion auf 0. Frame mit diameter=7 \dotfill  &   \pageref{fig:kap3_d=7}\\

locate()-Funktion auf 0. Frame mit diameter=5 \dotfill  &   \pageref{fig:kap3_d=5}\\

Ein Teil des initialen Dataframes \dotfill  &   \pageref{fig:kap3_initDataframe}\\

Ein Teil des sortierten Dataframes \dotfill  &   \pageref{fig:kap3_halbsortDataframes}\\

locate()-Funktion auf 0. Frame mit ’mimass=189.72805’ \dotfill  &  \pageref{fig:kap3_m=189}\\

Sortierte Tabelle \dotfill  &   \pageref{fig:kap3_sortDataframes}\\

locate()-Funktion auf 0. Frame mit ’mimass=197’ \dotfill  &  \pageref{fig:kap3_m=197}\\

locate()-Funktion auf 0. Frame mit ’mimass=204’ \dotfill  &  \pageref{fig:kap3_m=204}\\

locate()-Funktion auf 0. Frame mit ’mimass=210’ \dotfill  &   \pageref{fig:kap3_m=210}\\

locate()-Funktion auf 0. Frame mit ’mimass=212’ \dotfill  &   \pageref{fig:kap3_m=212}\\

locate()-Funktion auf 0. Frame mit ’separation=6.0’ \dotfill  &   \pageref{fig:kap3_sep=6}\\

locate()-Funktion auf 0. Frame mit ’separation=7.0’ \dotfill  &   \pageref{fig:kap3_sep=7}\\

locate()-Funktion auf 0. Frame mit ’separation=6.3’ \dotfill  &   \pageref{fig:kap3_sep=6.3}\\

Vergleich von Bild 1 und Bild 62 mit denselben Parameterwerten. \dotfill  &   \pageref{fig:kap3_comp_Bild_1vs_62}\\

Struktogramm der Methode.: get particles per image as array() \dotfill  &   \pageref{fig:kap3_strukto_part_per_array}\\

Vergleich von Bild 62 ohne und mit automatischer Optimierung der Parameter \dotfill  &   \pageref{fig:kap3_vergleich_Bild_62_mit_ohne}\\

Struktogramm der Methode: update frame() \dotfill & \pageref{fig:kap3_strukto_update_frame}\\

Frontend presentation \dotfill  &   \pageref{fig:kap_App/GUI presentation}\\
\end{tabularx}
