%\addcontentsline{toc}{chapter}{Ausblick}
\chapter{Ausblick}

Aus den Ergebnissen und dem derzeitigen Zustand unseres Tools, das uns nur erlaubt, Partikel zu erkennen, die Anzahl der Partikel im Laufe des Videos automatisch zu stabilisieren und die Ergebnisse angemessen anzuzeigen, ohne dabei die Änderung eines bestimmten Bildparameters zu vergessen, wird klar, dass es noch viele Dinge gibt, die wir verbessern und hinzufügen können. Leider hatten wir im Rahmen dieser Bachelorarbeit nicht die Zeit, dies zu tun. 
Dem Gedankengang zufolge sehen wir die möglichen Verbesserungen des Tools wie folgt: 
\begin{itemize}
 \item Die Einfügung einer Identifikation für jeden Partikel. So dass jedes Partikel einen ID-Code hat, der im Laufe der Zeit im Video nachverfolgt werden kann. 
 \item Die Bewegungsverläufe der einzelnen Partikel mithilfe dieser ID durch das gesamte Video verfolgen
 \item Erkennung von verschiedenen Ereignissen, die im Video stattfinden, ebenfalls anhand dieser ID. Genauer gesagt, Ereignisse der Spaltung (wenn sich ein Partikel in zwei oder mehr Partikel aufspaltet) und der Verschmelzung (wenn zwei oder mehr Partikel zusammenkommen, um ein einziges zu bilden) von Partikeln.
 \item Damit die Ergebnisse eine höhere Genauigkeit haben, wäre es interessant, sich mit dem 3D-Aspekt der Erkennung zu beschäftigen. Neben der horizontalen und vertikalen Position, die durch x bzw. y angegeben wird, gibt es noch eine dritte Position, die die z-Richtung bestimmt.
Dies würde die Zuverlässigkeit und Genauigkeit der Ergebnisse signifikant erhöhen.
\end{itemize}