%\addcontentsline{toc}{chapter}{Fazit}
\chapter{Fazit}
Das Ziel unserer Arbeit war es, ein Werkzeug (Software) zu entwickeln, mit dem man in einem bestimmten Video Bild für Bild die vorhandenen Partikel erkennen kann. Anschließend sollte die Anzahl der gefundenen Partikel automatisch konstant gehalten werden (z.B. um zu verhindern, dass durch Lichtveränderungen im Video zu viele Partikel entdeckt werden).\\ Danach sollte eine Benutzeroberfläche entwickelt werden, die es ermöglicht, die Ergebnisse aus den vorherigen Punkten optimal zu bewerten und zu beurteilen. Schließlich sollte auch die Möglichkeit bestehen, die Werte der Parameter, die zu den in den ersten Punkten erzielten Ergebnissen geführt haben, anzupassen, ohne die bereits validierten Ergebnisse zu verändern.\\
Im Kapitel \ref{kap1} haben wir zunächst eine vergleichende Studie verschiedener Werkzeuge durchgeführt, die die Möglichkeit bieten, das erste Ziel (Partikeldetektion) zu erreichen. Dies war \textbf{trackpy}.  Im Kapitel \ref{kap2} stellten wir trackpy vor und berichteten über unsere Erfahrungen als Nutzer.
Erst in Kapitel \ref{kap3} machten wir praktische Fortschritte, indem wir die zu verwendenden Parameter (siehe \ref{kap3_param_loacate}) vorstellten und einen systematischen Weg aufzeigten, wie man angemessene Werte für das erste Bild(Frame) finden kann (siehe \ref{kap3_OP}).\\
 Ebenfalls in diesem Kapitel, im Teil \ref{kap3_OP}, wird erklärt, wie die Lösung, die wir gefunden haben, um die Anzahl der Partikel in einem Intervall zu halten, funktioniert. Der Abschnitt \ref{kap3_bearb_einz_bild} seinerseits behandelt die Frage der Änderung von Parameterwerten und deren Anwendung in unserem Kontext.\\
Schließlich zeigen wir im Kapitel \ref{kap6} den Installationsprozess sowohl des Frontends als auch des Backends unseres Tools (Software). Wir haben auch die von uns entworfene Benutzeroberfläche vorgestellt, die unserer Meinung nach optimal für die Visualisierung und Bewertung der Parameterwerte ist, die zur Erzielung von Ergebnissen verwendet werden.


%\addcontentsline{toc}{section}{Fazit}
%\section{Fazit}

%\addcontentsline{toc}{section}{Eigene Meinung/Reflektion}
%\section{Eigene Meinung/Reflektion}