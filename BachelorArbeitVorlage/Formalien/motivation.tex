\chapter{Einleitung \label{Einleitung}}
Diese Arbeit befasst sich mit der Erkennung und Bewertung von Partikeln auf der Grundlage von Videodaten. Dazu werden die zu untersuchenden Videos in Sequenzen von Bildern segmentiert, durch die die Erkennung und dann die Auswertung Bild für Bild erfolgt. Dies geschieht unter Verwendung von Bibliotheken und/oder Frameworks, die für diesen Zweck konzipiert wurden.

\section{Motivation \label{Einleitung:Motivation}}
Jenseits eines besonderen Interesses an Biologie, Mikroskopie und Medizin im Allgemeinen gibt es auf medizinischer oder biologischer Seite einen stetig wachsenden Bedarf an immer leistungsfähigeren organischen Zellanalysen. Dies geschieht natürlich mit dem Ziel, die Funktionsweise (Reaktion) von Mikrozellen besser zu verstehen und dadurch die Qualität der Behandlung zu verbessern und/oder Prognosen zu erstellen.\\
In diesem Sinne haben wir uns überlegt, ein Werkzeug zu entwickeln, das als Sprungbrett für weitere Arbeiten genutzt werden kann, um so einen bescheidenen Beitrag zur Entwicklung der oben genannten Bereiche zu leisten.

\section{Ausgangssituation \label{Einleitung:Ausgangssituation}}
Diese Arbeit soll die erste in einer Reihe von zukünftigen Arbeiten sein. Deshalb ist der Ausgangspunkt hier nichts anderes als das Material, das zur Verfügung gestellt wird, um die Daten daraus zu extrahieren, nämlich die Videos.\\
Da es also keinen Vorgänger gibt, werden wir uns zunächst auf eine Art Vergleichsstudie verschiedener Tools konzentrieren, die uns helfen können, unsere Ziele zu erreichen (\textit{siehe Kapitel Bibliothek/Software}$^{\ref{kap1}}$). Anschließend werden wir die Ergebnisse dieser Studie verwenden, um mit dem eigentlichen Zweck der Arbeit fortzufahren, wie er in der \ref{Einleitung:Zielsetzung} beschrieben ist.

\section{Zielsetzung \label{Einleitung:Zielsetzung}}
Ziel dieser Bachelorarbeit ist es, wie bereits erwähnt, ein Werkzeug (Software) zu entwerfen, mit dem man Partikel in einem bestimmten Video aufspüren kann. Dabei wird die Anzahl der gefundenen Partikel in einem vorgegebenen Bereich gehalten.\\
Anschließend soll eine grafische Benutzeroberfläche entwickelt werden, die eine schnelle Auswertung der Ergebnisse ermöglicht.

\section{Abgrenzung \label{Einleitung:Abgrenzung}}
In Anbetracht des Rahmens unserer Arbeit ist es notwendig, an dieser Stelle zu erwähnen, dass die Erkennung und Bewertung von Partikeln auf der Grundlage von Videos eine Reihe von Faktoren nicht einschließt. Dies bezieht sich unter anderem auf die Interpretation der Ergebnisse. Dabei handelt es sich sowohl um die Position der Partikel als auch um deren Identifizierung.\\
Diese Arbeit wird daher zunächst nur dazu dienen, die Partikel Bild für Bild zu erkennen. Anschließend werden die Bilder mit den erkannten Partikeln mit Hilfe einer eigens dafür entwickelten Schnittstelle visualisiert.\\ Schließlich können wir Parameteränderungen auf ein oder mehrere spezifische Bilder anwenden, ohne die bereits validierten Bilder zu verändern.



